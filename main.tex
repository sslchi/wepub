\documentclass{wepub}

\usepackage{lipsum}
\usepackage{xeCJK,graphicx}

\title{微信公众号\LaTeX 排版}
\date{2019年11月30日}
\author{王可达}

\begin{document}
	\maketitle

\begin{weaxiom}
	最近开始写一个关于数学的公众号,里面有很多数学公式,使用各类公众号排版工具都无法高效的完成排版,于是就自己写了一个\LaTeX 模板,用来进行公众号排版。具体做法是,先生成PDF文档,然后用手机截成长图,以图片形式上传到公众号。文档的尺寸和字体的大小是根据手机屏幕的大小量身定做的,以尽量保证文档在手机上仍然可以看着比较舒服。
\end{weaxiom}
在模板里,主要定义了三个环境,可以使用
\begin{verbatim*}
\begin{wethm}
	这里是文字
\end{wethm}
\begin{welemma}
这里是文字
\end{welemma}
\begin{weaxiom}
这里是文字
\end{weaxiom}
\end{verbatim*}
进行使用,效果如下:
\begin{wethm}
	这里是文字
\end{wethm}
\begin{welemma}
	这里是文字
\end{welemma}
\begin{weaxiom}
	这里是文字
\end{weaxiom}
也可以使用公式或者图片环境
\begin{verbatim*}
\begin{equation}
a^2+b^2=c^2 \notag
\end{equation}

\begin{figure}[htbp]\caption{扫码关注我的公众号}
	\includegraphics[width=\linewidth]{logo.jpg}
\end{figure}
\end{verbatim*}
插入图片,效果如下:
\begin{equation}
a^2+b^2=c^2 \notag
\end{equation}

\begin{figure}[htbp]
	\includegraphics[width=\linewidth]{logo.jpg}\caption{扫码关注我的公众号}
\end{figure}
关注我的公众号,回复wepub获取模板。
\end{document}